\documentclass{ecnreport}

\stud{CORO / Option Robotique}
\topic{Robot Operating System}
\author{G. Garcia, O. Kermorgant}



\newcommand{\ttt}[1]
{
{\tt{#1}}
}


\begin{document}

\inserttitle{Robot Operating System}

\section*{Content of the labs}

The goal of these labs is to discover several of the aspects of using and programming with ROS. 
There are 4 labs that will deal about increasing complexity of the ROS ecosystem. Most of the labs can be done on your 
own computer, assuming that you have a working ROS installation. 

\subsection*{Setting up the computer}

Computers from the lab room at ECN should be already set up. If you want to install ROS on your own,
you can find several ways to do so, depending on your motivation:
\begin{itemize}
 \item Go to \url{www.ros.org} and follow the tutorials by your own.
 \item Download the Virtual Machine at \url{http://www.irccyn.ec-nantes.fr/\~kermorga/coding\%20tools.html\#vm}. This VM corresponds
 more or less to the ECN computers. 
 \item If you have an Ubuntu, go to \url{http://www.irccyn.ec-nantes.fr/\~kermorga/coding%20tools.html\#scripts} where you 
 can find several helper scripts to install ROS, setup your profile and install V-REP, a simulator
 that is used for some labs. 
\end{itemize}

At ECN, we assume the \ttt{ROS} workspace is located at \ttt{\textasciitilde/ros}. Most of the official tutorials use \ttt{\textasciitilde/catkin\_ws} or something similar
but we think a simple \ttt{ros} folder is quite explicit. 
The directory tree is thus:
\begin{center}\begin{minipage}{.4\linewidth}
 \dirtree{%
.1 home.
.2 Documents.
.2 Downloads.
.2 ros.
.3 build.
.3 devel.
.3 src.
.4 some\_package.
.5 CMakeLists.txt.
.4 some\_other\_package.
.5 CMakeLists.txt.
} \end{minipage}
\end{center}

If you decide to use another workspace path, adapt the command lines accordingly.


\subsection*{How to compile}

The main tool used to compile ROS packages is called \ttt{catkin}. Basically, calling \ttt{catkin build} will crawl through
all your packages and resolve the dependency tree. It will then compile first the independant packages, then the others. 
The actual compilation is done through \ttt{cmake} and \ttt{make} but {\tt{catkin}} use them automatically.

\begin{center}\fbox{\parbox{\textwidth}{%
 You have to call \ttt{catkin build} each time you have a new package, either because you downloaded one in your workspace
 or because you manually created one.}}
\end{center}


\subsection*{How to use QtCreator with ROS}

As we saw during C++ labs, an IDE such as Qt Creator is pretty useful. For ROS it will be even more as ROS uses many classes and the IDE will give you autocompletion 
and hints for their methods. 
A guide on how to use Qt Creator with ROS is available at \url{http://www.irccyn.ec-nantes.fr/~kermorga/coding%20tools.html#config}.\\

Do not forget to launch Qt Creator from the command line (\ttt{qtcreator}) or use the desktop shortcut that simulates a console launch.\\

Once \ttt{catkin build} is called, remember that the \ttt{build} directory of a given package called \ttt{package\_name} is at \ttt{\textasciitilde/ros/build/package\_name}.\\

ECN computers and the Virtual Machine are configured to compile in Debug mode by default, which helps the bug hunt.

\newpage

\section{Lab 1 - launchfiles, remaps and gui tools}

\section{Lab 2 - package creation, dependencies and topics}

\section{Lab 3 - transforms and services}

\section{Lab 4 - working with images}






\end{document}
